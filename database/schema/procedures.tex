\subsection{atendimentos\_proximos}

\textbf{In:} k (página desejada, a primeira é k = 0)

\textbf{Out:} Nada

Uma tabela, ordenada por horario-agendado, com as colunas:

\begin{itemize}
	\item nome
	\item numero\_do\_telefone
	\item tipo\_de\_atendimento
	\item horario\_agendado
	\item atendimento\_id
\end{itemize}

\begin{verbatim}
call atendimentos_proximos(0)
\end{verbatim}


\subsection{tratamentos\_cronicos\_proximos\_de\_estourar}

\textbf{In:} k (página desejada, a primeira é k = 0)

\textbf{Out:} Nada

\textbf{Retorno:} Uma tabela com as colunas:

\begin{itemize}
	\item nome\_do\_cliente
	\item tipo\_de\_tratamento
	\item numero\_do\_telefone
	\item tratamento\_id
\end{itemize}

\begin{verbatim}
call tratamentos_cronicos_proximos_de_estourar(0)
\end{verbatim}

% página 2

\subsection{dados\_clientes\_idade}


\textbf{In:} k (página desejada, a primeira é k = 0)

\textbf{Out:} Media, mediana, desvio padrão

\textbf{Retorno:} Uma tabela com as colunas x e y, para plotar a linha

\begin{verbatim}
call dados_clientes_idade(0)
\end{verbatim}

\subsection{dados\_clientes\_estado\_civil}


\textbf{In:} Nada

\textbf{Out:} Nada

\textbf{Retorno:} Uma tabela com as colunas:

\begin{itemize}
	\item estado\_civil
	\item quant
\end{itemize}

\begin{verbatim}
	call dados_clientes_estado_civil()
\end{verbatim}




% página 3

\subsection{medicos\_funcionarios\_vs\_prestadores}

\textbf{In:} Nada

\textbf{Out:} Nada\subsection{lista\_de\_clientes}

\textbf{In:} k (página desejada, a primeira é k = 0), filtro

\textbf{Out:} Nada

\textbf{Retorno:} Uma tabela com as colunas:

\begin{itemize}
	\item nome
	\item idade
	\item sexo
	\item numero\_do\_telefone
	\item cpf
\end{itemize}

\begin{verbatim}
	lista_de_clientes(0,"")
\end{verbatim}

\textbf{Retorno:} Uma tabela com as colunas:

\begin{itemize}
	\item total
	\item funcionarios
	\item prestadores
\end{itemize}

\begin{verbatim}
medicos_funcionarios_vs_prestadores()
\end{verbatim}

\subsection{dados\_das\_especialidades\_quant\_medicos}

\textbf{In:} Nada

\textbf{Out:} media, desvio padrão, máximo, minimo

\textbf{Retorno:} três tabelas:

\begin{itemize}
	\item nomes das especialidades com valores máximo
	\subitem nome
	\item nomes das especialidades com valores mínimo
	\subitem nome
	\item informações para gráfico de barra
	\subitem nome
	\subitem valor
\end{itemize}

\begin{verbatim}
call dados_das_especialidades_quant_medicos(@media, @desvio, @maximo, @minimo);

select @media, @desvio, @maximo, @minimo from dual;
\end{verbatim}


\subsection{dados\_das\_especialidades\_valores\_arrecadados}

\textbf{In:} Nada

\textbf{Out:} media, desvio padrão, máximo, minimo

\textbf{Retorno:} três tabelas:

\begin{itemize}
	\item nomes das especialidades com valores máximo
	\subitem nome
	\item nomes das especialidades com valores mínimo
	\subitem nome
	\item informações para gráfico de barra
	\subitem nome
	\subitem valor
\end{itemize}

\begin{verbatim}
call dados_das_especialidades_valores_arrecadados(@media, @desvio, @maximo, @minimo);

select @media, @desvio, @maximo, @minimo from dual;
\end{verbatim}


\subsection{dados\_das\_especialidades\_quant\_atendimentos}

\textbf{In:} Nada

\textbf{Out:} media, desvio padrão, máximo, minimo

\textbf{Retorno:} três tabelas:

\begin{itemize}
	\item nomes das especialidades com valores máximo
	\subitem nome
	\item nomes das especialidades com valores mínimo
	\subitem nome
	\item informações para gráfico de barra
	\subitem nome
	\subitem valor
\end{itemize}

\begin{verbatim}
call dados_das_especialidades_quant_atendimentos(@media, @desvio, @maximo, @minimo);

select @media, @desvio, @maximo, @minimo from dual;
\end{verbatim}


\subsection{dados\_medicos\_vs\_quant\_pacientes}

\textbf{In:} Nada

\textbf{Out:} media, desvio padrão, máximo, minimo

\textbf{Retorno:} três tabelas:

\begin{itemize}
	\item nomes das especialidades com valores máximo
	\subitem nome
	\item nomes das especialidades com valores mínimo
	\subitem nome
	\item informações para gráfico o pseudo histograma
	\subitem nome
	\subitem valor
\end{itemize}

\begin{verbatim}
	call dados_medicos_vs_quant_pacientes(@media, @desvio, @maximo, @minimo);
	
	select @media, @desvio, @maximo, @minimo from dual;
\end{verbatim}


\subsection{lista\_de\_medicos}

\textbf{In:} k (página desejada, a primeira é k = 0), filtro

\textbf{Out:} Nada

\textbf{Retorno:} Uma tabela com as colunas:

\begin{itemize}
	\item nome
	\item idade
	\item crm
	\item sexo
	\item numero\_do\_telefone
\end{itemize}

\begin{verbatim}
lista_de_medicos(0,"")
\end{verbatim}

\subsection{lista\_de\_medicos\_prestadores}

\textbf{In:} k (página desejada, a primeira é k = 0), filtro

\textbf{Out:} Nada

\textbf{Retorno:} Uma tabela com as colunas:

\begin{itemize}
	\item nome
	\item idade
	\item crm
	\item sexo
	\item numero\_do\_telefone
\end{itemize}

\begin{verbatim}
	lista_de_medicos\_prestadores(0,"")
\end{verbatim}

\subsection{lista\_de\_medicos\_funcionarios}

\textbf{In:} k (página desejada, a primeira é k = 0), filtro

\textbf{Out:} Nada

\textbf{Retorno:} Uma tabela com as colunas:

\begin{itemize}
	\item nome
	\item idade
	\item crm
	\item sexo
	\item numero\_do\_telefone\_funcionarios
\end{itemize}

\begin{verbatim}
	lista_de_medicos(0,"")
\end{verbatim}

\subsection{dados\_funcionarios\_estado\_civil}


\textbf{In:} Nada

\textbf{Out:} Nada

\textbf{Retorno:} Uma tabela com as colunas:

\begin{itemize}
	\item estado\_civil
	\item quant
\end{itemize}

\begin{verbatim}
	call dados_funcionarios_estado_civil()
\end{verbatim}

\subsection{dados\_funcionarios\_idade}


\textbf{In:} k (página desejada, a primeira é k = 0)

\textbf{Out:} media, mediana, desvio padrao

\textbf{Retorno:} Uma tabela com as colunas x e y, para plotar a linha

\begin{verbatim}
call dados_funcionarios_idade(@media, @mediana, @desvio);
select @media, @mediana, @desvio from dual;
\end{verbatim}


\subsection{lista\_de\_funcionarios}

\textbf{In:} k (página desejada, a primeira é k = 0), filtro

\textbf{Out:} Nada

\textbf{Retorno:} Uma tabela com as colunas:

\begin{itemize}
	\item nome
	\item idade
	\item sexo
	\item numero\_do\_telefone
	\item cpf
\end{itemize}

\begin{verbatim}
	lista_de_funcionarios(0,"")
\end{verbatim}


\subsection{dados\_plano\_de\_saude\_quantidade\_cliente}

\textbf{In:} Nada

\textbf{Out:} media, desvio padrão, máximo, minimo

\textbf{Retorno:} três tabelas:

\begin{itemize}
	\item nomes das especialidades com valores máximo
	\subitem nome
	\item nomes das especialidades com valores mínimo
	\subitem nome
	\item informações para pseudo histograma
	\subitem x
	\subitem y
\end{itemize}

\begin{verbatim}
	call dados_plano_de_saude_quantidade_cliente(@media, @desvio, @maximo, @minimo);
	
	select @media, @desvio, @maximo, @minimo from dual;
\end{verbatim}

\subsection{dados\_plano\_de\_saude\_atendimentos\_aceitos}

\textbf{In:} Nada

\textbf{Out:} media, desvio padrão, máximo, minimo

\textbf{Retorno:} três tabelas:

\begin{itemize}
	\item nomes das especialidades com valores máximo
	\subitem nome
	\item nomes das especialidades com valores mínimo
	\subitem nome
	\item informações para gráfico de barra
	\subitem x
	\subitem y
\end{itemize}

\begin{verbatim}
	call dados_plano_de_saude_atendimentos_aceitos(@media, @desvio, @maximo, @minimo);
	
	select @media, @desvio, @maximo, @minimo from dual;
\end{verbatim}

\subsection{dados\_plano\_de\_saude\_quantidade\_de\_medicos}

\textbf{In:} Nada

\textbf{Out:} media, desvio padrão, máximo, minimo

\textbf{Retorno:} três tabelas:

\begin{itemize}
	\item nomes das especialidades com valores máximo
	\subitem nome
	\item nomes das especialidades com valores mínimo
	\subitem nome
	\item informações para gráfico de barra
	\subitem x
	\subitem y
\end{itemize}

\begin{verbatim}
	call dados_plano_de_saude_quantidade_de_medicos(@media, @desvio, @maximo, @minimo);
	
	select @media, @desvio, @maximo, @minimo from dual;
\end{verbatim}

\subsection{dados\_plano\_de\_saude\_vs\_desconto}

\textbf{In:} Nada

\textbf{Out:} media, desvio padrão, máximo, minimo

\textbf{Retorno:} três tabelas:

\begin{itemize}
	\item nomes das especialidades com valores máximo
	\subitem nome
	\item nomes das especialidades com valores mínimo
	\subitem nome
	\item informações para gráfico de barra
	\subitem x
	\subitem y
\end{itemize}

\begin{verbatim}
	call dados_plano_de_saude_vs_desconto(@media, @desvio, @maximo, @minimo);
	
	select @media, @desvio, @maximo, @minimo from dual;
\end{verbatim}

\subsection{lista\_de\_planos}

\textbf{In:} k (página desejada, a primeira é k = 0), filtro

\textbf{Out:} Nada

\textbf{Retorno:} Uma tabela com as colunas:

\begin{itemize}
	\item nome
	\item plano\_de\_saude\_id
\end{itemize}

\begin{verbatim}
	lista_de_planos(0,"")
\end{verbatim}

\subsection{obter\_dados\_tratamentos}

\textbf{In:} Nada

\textbf{Out:} Nada

\textbf{Retorno:} 2 tabelas:

\begin{itemize}
	\item quantidade agrupado por ser cronico (pizza)
	\subitem eh\_cronico
	\subitem quant
	
	\item quantidade agrupado por tipo de tratamento (barra)
	\subitem nome
	\subitem quant
\end{itemize}

\subsection{lista\_de\_tratamentos}

\textbf{In:} k (página desejada, a primeira é k = 0), filtro

\textbf{Out:} Nada

\textbf{Retorno:} Uma tabela com as colunas:

\begin{itemize}
	\item tratamento\_id
	\item cliente\_nome
	\item tipo\_de\_tratamento
	\item quant\_de\_atendimentos
\end{itemize}

\begin{verbatim}
	lista_de_tratamentos(0,"")
\end{verbatim}

\subsection{obter\_dados\_atendimentos\_realizados}
\textbf{In:} data de inicio, data do fim

\textbf{Out:} Nada

\textbf{Retorno:} Uma tabela com as colunas:

\begin{itemize}
	\item valor\_recebido\_media
	\item valor\_recebido\_desvio
	\item valor\_recebido\_soma
	\item valor\_arrecadado\_media
	\item valor\_arrecadado\_desvio
	\item valor\_arrecadado\_total
	\item quantidade
\end{itemize}

\begin{verbatim}
	call obter_dados_atendimentos_realizados("2021-5-1", "2023-06-19");
\end{verbatim}

\subsection{lista\_de\_atendimentos\_realizados}

\textbf{In:} k (página desejada, a primeira é k = 0), filtro, data de inicio, data do fim

\textbf{Out:} Nada

\textbf{Retorno:} Uma tabela com as colunas:

\begin{itemize}
	\item nome\_cliente
	\item tipo\_de\_atendimento
	\item nome\_medico
	\item horario\_inicio\_real
	\item atendimento\_id
\end{itemize}

\begin{verbatim}
	call lista_de_atendimentos_realizados(1, "be", "2021-5-1", "2023-06-19")
\end{verbatim}

\subsection{obter\_dados\_pessoais}

\textbf{In:} cpf

\textbf{Out:} Nada

\textbf{Retorno:} Uma tabela com as colunas:

\begin{itemize}
	\item cpf
	\item nome
	\item estado\_civil
	\item sexo
	\item data\_de\_nascimento
	\item idade
	\item cep
	\item estado
	\item cidade
	\item bairro
	\item logradouro
	\item numero
	\item complemento
	\item email
	\item numero\_do\_telefone
\end{itemize}

\begin{verbatim}
call obter_dados_pessoais(329284240)
\end{verbatim}


\subsection{atendimentos\_realizados\_por\_cliente\_dados}

\textbf{In:} crm

\textbf{Out:} Nada

\textbf{Retorno:} Uma tabela com as colunas:

\begin{itemize}
	\item quant
	\item data\_ultima\_consulta
\end{itemize}

\begin{verbatim}
call atendimentos_realizados_por_cliente_dados(50545756421);
\end{verbatim}

\subsection{atendimentos\_realizados\_por\_um\_cliente\_lista}

\textbf{In:} k (página desejada, a primeira é k = 0), cpf, filtro

\textbf{Out:} Nada

\textbf{Retorno:} Uma tabela com as colunas:

\begin{itemize}
	\item cpf
	\item nome\_medico
	\item sala
	\item horario\_inicio\_real
	\item atendimento\_id
\end{itemize}

\begin{verbatim}
call atendimentos_realizados_por_um_cliente_lista(0,50545756421, "")
\end{verbatim}

\subsection{atendimentos\_agendados\_por\_cliente\_dados}

\textbf{In:} crm

\textbf{Out:} Nada

\textbf{Retorno:} Uma tabela com as colunas:

\begin{itemize}
	\item quant
	\item data\_proxima\_consulta
\end{itemize}

\begin{verbatim}
	call atendimentos_agendados_por_cliente_dados(50545756421);
\end{verbatim}

\subsection{atendimentos\_agendados\_por\_um\_cliente\_lista}

\textbf{In:} k (página desejada, a primeira é k = 0), cpf, filtro

\textbf{Out:} Nada

\textbf{Retorno:} Uma tabela com as colunas:

\begin{itemize}
	\item cpf
	\item nome\_medico
	\item sala
	\item horario\_agendado
	\item atendimento\_id
\end{itemize}

\begin{verbatim}
	call atendimentos_agendados_por_um_cliente_lista(0,50545756421, "")
\end{verbatim}

\subsection{tratamentos\_por\_um\_cliente\_lista}

\textbf{In:} k (página desejada, a primeira é k = 0), cpf

\textbf{Out:} Nada

\textbf{Retorno:} Uma tabela com as colunas:

\begin{itemize}
	\item nome
	\item quant\_consultas
	\item tratamento\_id
\end{itemize}

\begin{verbatim}
	call atendimentos_agendados_por_um_cliente_lista(0,50545756421, "")
\end{verbatim}

\subsection{planos\_de\_saude\_de\_um\_cliente\_lista}

\textbf{In:} k (página desejada, a primeira é k = 0), cpf

\textbf{Out:} Nada

\textbf{Retorno:} Uma tabela com as colunas:

\begin{itemize}
	\item nome
	\item quant\_realizadas
	\item quant\_agendados
\end{itemize}

\begin{verbatim}
	call planos_de_saude_de_um_cliente_lista(0,50545756421)
\end{verbatim}


\subsection{planos\_de\_saude\_de\_um\_cliente\_lista}

\textbf{In:} Nada

\textbf{Out:} Nada

\textbf{Retorno:} Uma tabela com as colunas:

\begin{itemize}
	\item nome
\end{itemize}

\begin{verbatim}
	call lista_de_doencas_pre_existentes(50545756421)
\end{verbatim}


\subsection{depedentes\_de\_um\_funcionario}

\textbf{In:} k (página desejada, a primeira é k = 0), cpf

\textbf{Out:} Nada

\textbf{Retorno:} Uma tabela com as colunas:

\begin{itemize}
	\item nome
	\item cpf
\end{itemize}

\begin{verbatim}
	call depedentes_de_um_funcionario(0,50545756421)
\end{verbatim}

\subsection{informacoes\_escolares\_de\_um\_medico}

\textbf{In:} k (página desejada, a primeira é k = 0), cpf

\textbf{Out:} Nada

\textbf{Retorno:} Uma tabela com as colunas:

\begin{itemize}
	\item escola\_de\_origem
	\item tipo\_de\_residencia\_medica
\end{itemize}

\begin{verbatim}
	call informacoes_escolares_de_um_medico(3)
\end{verbatim}

\subsection{especialidades\_de\_um\_medico}

\textbf{In:} k (página desejada, a primeira é k = 0), crm

\textbf{Out:} nada

\textbf{Retorno:} Uma tabela com as colunas:

\begin{itemize}
	\item nome
\end{itemize}

\begin{verbatim}
	call informacoes_escolares_de_um_medico(3)
\end{verbatim}

\subsection{planos\_de\_saude\_de\_um\_medico}

\textbf{In:} k (página desejada, a primeira é k = 0), crm

\textbf{Out:} nada

\textbf{Retorno:} Uma tabela com as colunas:

\begin{itemize}
	\item nome
\end{itemize}

\begin{verbatim}
	call planos_de_saude_de_um_medico(3)
\end{verbatim}

\subsection{atendimentos\_realizados\_por\_medico\_dados}

\textbf{In:} crm

\textbf{Out:} nada

\textbf{Retorno:} Uma tabela com as colunas:

\begin{itemize}
	\item quant
	\item data\_ultima\_consulta
\end{itemize}

\begin{verbatim}
	call atendimentos_realizados_por_medico_dados(3)
\end{verbatim}


\subsection{atendimentos\_realizados\_por\_um\_medico\_lista}

\textbf{In:} k, crm, filtro

\textbf{Out:} nada

\textbf{Retorno:} Uma tabela com as colunas:

\begin{itemize}
	\item cpf
	\item nome\_cliente
	\item sala
	\item horario\_inicio\_real
	\item valor\_recebido\_por\_medico
	\item atendimento\_id
\end{itemize}

\begin{verbatim}
call atendimentos_realizados_por_um_medico_lista(0, 48,"")
\end{verbatim}

\subsection{atendimentos\_agendados\_por\_medico\_dados}

\textbf{In:} crm

\textbf{Out:} nada

\textbf{Retorno:} Uma tabela com as colunas:

\begin{itemize}
	\item quant
	\item data\_proxima\_consulta
\end{itemize}

\begin{verbatim}
	call atendimentos_realizados_por_um_medico_lista(0, 48,"")
\end{verbatim}


\subsection{atendimentos\_agendados\_por\_um\_medico\_lista}

\textbf{In:} k, crm, filtro

\textbf{Out:} nada

\textbf{Retorno:} Uma tabela com as colunas:

\begin{itemize}
	\item cpf
	\item nome\_cliente
	\item sala
	\item horario\_agendado
	\item valor\_recebido\_por\_medico
	\item atendimento\_id
\end{itemize}

\begin{verbatim}
	call atendimentos_agendados_por_um_medico_lista(0, 48,"")
\end{verbatim}

\subsection{dados\_sobre\_atendimento\_realizado}

\textbf{In:} atendimento\_id

\textbf{Out:} nada

\textbf{Retorno:} Uma tabela com as colunas:

\begin{itemize}
\item horario\_agendado
\item horario\_inicio\_real
\item horario\_fim\_real
\item nome\_cliente
\item nome\_medico
\item valor
\item comissao\_da\_clinica
\item valor\_recebidos\_por\_medico
\item data\_de\_recebimento
\item data\_do\_reparsse\_ao\_medico
\item tipo\_de\_atendimentos
\end{itemize}

\begin{verbatim}
select obter_n_atendimentos_agendados _por_um_medico(3) from dual;
\end{verbatim}


\subsection{cadastrar\_pessoa}

\textbf{In:} um json (olhar exemplo)

\textbf{Out:} nada

\textbf{Retorno:} nada

\begin{verbatim}
call cadastrar_pessoa('{
	"CPF" : 84958640029,
	"nome" : "insira um nome verdadeiro aqui",
	"estado_civil" : "Solteiro(a)",
	"sexo" : 0,
	"data_de_nascimento" : "1939-01-05",
	"numero_do_telefone" : " 32606-8312",
	"cep" : "25721705",
	"estado" : "MT",
	"cidade" : "Santo Antônio do Içá",
	"bairro" : "Cangaíba",
	"logradouro" : "Rua José Paulino",
	"complemento" : 216,
	"numero" : 81,
	"email" : "davikristiansen@gmail.com"
}');
\end{verbatim}

\subsection{obter\_funcoes}

\textbf{In:} nada

\textbf{Out:} nada

\textbf{Retorno:} uma tabela:

\begin{itemize}
	\item funcao\_id
	\item nome
\end{itemize}

\subsection{cadastrar\_funcionario}
\textbf{In:} um json (olhar exemplo)

\textbf{Out:} nada

\textbf{Retorno:} nada

\begin{verbatim}
	call cadastrar_funcionario('{
		"salario_bruto" : 127745,
		"CPF" : 62992881594,
		"funcao_id" : 4,
		"depedentes": [
			26115330408,
			38939575024,
			92289773450
		]}');
\end{verbatim}

\subsection{cadastrar\_medico}
\textbf{In:} um json (olhar exemplo)

\textbf{Out:} nada

\textbf{Retorno:} nada

\begin{verbatim}
call cadastrar_medico('{
	"CRM" : 60,
	"escola_de_origem" : "Universidade Federal De Minas Gerais (UFMG)",
	"tipo_de_residencia_medica_id" : 9,
	"especialidades": [1, 4, 10],
	"planos_de_saude": [1, 6, 10]
}');
\end{verbatim}

\subsection{medico\_funcionario}
\textbf{In:} crm, cpf

\textbf{Out:} nada

\textbf{Retorno:} nada

\begin{verbatim}
	call medico_funcionario();
\end{verbatim}

\subsection{cadastrar\_medico\_prestador\_de\_servico}
\textbf{In:} crm, cpf

\textbf{Out:} nada

\textbf{Retorno:} nada

\begin{verbatim}
	call cadastrar_medico_prestador_de_servico();
\end{verbatim}

\subsection{obter\_tipos\_de\_residencia\_medica}

\textbf{In:} nada

\textbf{Out:} nada

\textbf{Retorno:} uma tabela:

\begin{itemize}
	\item tipo\_de\_residencia\_medica\_id
	\item nome
\end{itemize}

\begin{verbatim}
call obter_tipos_de_residencia_medica();
\end{verbatim}

\subsection{obter\_especialidades}

\textbf{In:} nada

\textbf{Out:} nada

\textbf{Retorno:} uma tabela:

\begin{itemize}
	\item especialidade\_id
	\item nome
\end{itemize}

\begin{verbatim}
call obter_especialidades();
\end{verbatim}

\subsection{obter\_plano\_de\_saude}

\textbf{In:} nada

\textbf{Out:} nada

\textbf{Retorno:} uma tabela:

\begin{verbatim}
call obter_plano_de_saude
\end{verbatim}

\begin{itemize}
	\item plano\_de\_saude\_id
	\item nome
\end{itemize}

\subsection{obter\_doencas}

\textbf{In:} nada

\textbf{Out:} nada

\textbf{Retorno:} uma tabela:

\begin{itemize}
	\item doenca\_id
	\item nome
\end{itemize}

\begin{verbatim}
\call obter_doencas
\end{verbatim}

\subsection{cadastrar\_cliente}
\textbf{In:} um json (olhar exemplo)

\textbf{Out:} nada

\textbf{Retorno:} nada

\begin{verbatim}
	call cadastrar_cliente('{
		"CPF" : 84958640029,
		"doencas_pre_existentes": [1, 4, 6],
		"planos_de_saude": [1, 6, 10]
	}');
\end{verbatim}

\subsection{obter\_lista\_completa\_tratamentos\_do\_cliente}
\textbf{In:} cpf

\textbf{Out:} nada

\textbf{Retorno:} tabela

\begin{itemize}
	\item nome
\end{itemize}

\subsection{obter\_tipos\_de\_atendimento}
\textbf{In:} nada

\textbf{Out:} nada

\textbf{Retorno:} tabela

\begin{itemize}
	\item nome
\end{itemize}

\subsection{obter\_planos\_de\_saude\_validos}
\textbf{In:} cpf, tipo\_de\_atendimento\_id

\textbf{Out:} nada

\textbf{Retorno:} tabela

\begin{itemize}
	\item nome
\end{itemize}

\subsection{obter\_medicos\_validos}
\textbf{In:} plano\_de\_saude\_id, tipo\_de\_atendimento\_id

\textbf{Out:} nada

\textbf{Retorno:} tabela

\begin{itemize}
	\item nome
\end{itemize}

\subsection{obter\_tipos\_de\_tratamento}

\textbf{In:} nada

\textbf{Out:} nada

\textbf{Retorno:} uma tabela:

\begin{itemize}
	\item nome
\end{itemize}

\subsection{agendar\_atendimento}

\textbf{In:} um json (veja o exemplo)

\textbf{Out:} nada

\textbf{Retorno:} nada

\begin{verbatim}
call agendar_atendimento('{
	"sala" : "A3",
	"horario_agendado" : "2020-06-19 08:30:00",
	"estado" : 0,
	"crm" : 2,
	"plano_de_saude_id" : 12,
	"tipo_de_atendimento_id" : 16,
	"tratamento_id" : 1
}');
\end{verbatim}

\subsection{criar\_tratamento}

\textbf{In:} tipo\_de\_tratamento, cpf

\textbf{Out:} nada

\textbf{Retorno:} nada
